\mbox{\hyperlink{index}{Main Page}} (Up) ~ ~ ~ ~ \mbox{\hyperlink{install_page}{Installation}} (Next) ~\newline
 ~\newline


P\+S\+C\+F+ is a software package for solving the polymer self-\/consistent field (S\+CF) theory in continuum. It is based on the nice G\+PU framework of \href{https://github.com/dmorse/pscfpp}{\texttt{ P\+S\+CF}} (which is only for the \char`\"{}standard\char`\"{} or known as the Edwards-\/\+Helfand model, {\itshape i.\+e.}, incompressible melts of continuous Gaussian chains with the Dirac {\itshape {$\delta$}}-\/function interactions, commonly used in polymer field theories) originally developed by Prof. David Morse and co-\/workers, but is improved with better numerical methods, less G\+PU memory usage and more flexible algorithms, and is extended to various discrete-\/chain models. Similar to the C++/\+C\+U\+DA version of P\+S\+CF, P\+S\+C\+F+ described here is written primarily in C++ with G\+PU accelerated code in C\+U\+DA.

Same as the C++/\+C\+U\+DA version of P\+S\+CF, P\+S\+C\+F+ is applicable to mixtures containing arbitrary acyclic copolymers, and preserves all of the nice features already implemented in the former, including the use of \href{https://docs.nvidia.com/cuda/cufft/index.html}{\texttt{ cu\+F\+FT}} on G\+PU, the use of Anderson mixing (which is performed on G\+PU) combined with a variable-\/cell scheme to simultaneously solve the nonlinear S\+CF equations and find the bulk periodicity for the ordered phases formed by block copolymer self-\/assembly (which speeds-\/up the calculation by about one order of magnitude), and the documentation produced by \href{https://www.doxygen.nl/}{\texttt{ Doxygen}}. Their differences and expected advantages of the latter include\+: 
\begin{DoxyItemize}
\item P\+S\+CF is only applicable to the \char`\"{}standard\char`\"{} model, while P\+S\+C\+F+ can also be applied to various discrete-\/chain models with finite-\/range non-\/bonded interactions commonly used in molecular simulations, thus providing the mean-\/field reference results for such simulations; see \href{https://github.com/qwcsu/PSCFplus/blob/master/doc/notes/Models.pdf}{\texttt{ Models.\+pdf}} for more details.  
\item For the continuous-\/\+Gaussian-\/chain models, P\+S\+C\+F+ uses the Richardson-\/extrapolated pseudo-\/spectral methods (denoted by R\+E\+P\+S-\/{\itshape K} with {\itshape K}=0,1,2,3,4) to solve the modified diffusion equations (which is the crux of S\+CF calculations of such models), while P\+S\+CF uses only R\+E\+P\+S-\/1. A larger {\itshape K}-\/value gives more accurate result at larger computational cost; see \href{https://github.com/qwcsu/PSCFplus/blob/master/doc/notes/REPS.pdf}{\texttt{ R\+E\+P\+S.\+pdf}} for more details.  
\item For 3D spatially periodic ordered phases such as those formed by block copolymer self-\/assembly, while P\+S\+CF uses fast Fourier transforms (F\+F\+Ts) between a uniform grid in the real space and that in the reciprocal space, for the Pmmm supergroup P\+S\+C\+F+ uses discrete cosine transforms instead of F\+F\+Ts to take advantage of the (partial) symmetry of an ordered phase to reduce the number of grid points, thus both speeding up the calculation and reducing the memory usage; see \href{https://pubs.acs.org/doi/10.1021/acs.macromol.0c01974}{\texttt{ {\itshape Qiang and Li}, {\bfseries{Macromolecules 53}}, 9943 (2020)}} for more details.  
\item In S\+CF calculations the (one-\/end-\/integrated) forward and backward propagators $q$ and $q^{\dagger}$ of each block usually take the largest memory usage, but the G\+PU memory is rather limited. While in P\+S\+CF the size of these propagators is $MN_s$, where $M$ denotes the number of grid points in real space and $N_s$ the number of contour discretization points on a continuous Gaussian chain (or the number of segments on a discrete chain), in P\+S\+C\+F+ the \char`\"{}slice\char`\"{} algorithm proposed by Li and Qiang can be used to reduce the size of $q$ to $M\sqrt{N_s}$ and that of $q^{\dagger}$ to just $M$, thus greatly reducing the G\+PU memory usage at the cost of computing $q$ twice; see \href{https://github.com/qwcsu/PSCFplus/blob/master/doc/notes/SavMem.pdf}{\texttt{ Sav\+Mem.\+pdf}} for more details.  
\item Since S\+CF equations are highly nonlinear, having a good initial guess is very important in practice as it determines not only which final solution (corresponding to a phase in block copolymer self-\/assembly) can be obtained but also how many iteration steps the solver ({\itshape e.\+g.}, the Anderson mixing) takes to converge these equations. P\+S\+C\+F+ uses automated calculation along a path (A\+C\+AP), where the converged solution at a neighboring point is taken as the initial guess at the current point in the parameter space. While this is similar to the \char`\"{}\+S\+W\+E\+E\+P\char`\"{} command in P\+S\+CF, the key for A\+C\+AP to be successful and efficient is that it automatically adjusts the step size along the path connecting the two points. In P\+S\+C\+F+, A\+C\+AP is further combined with the phase-\/boundary calculation between two specified phases, making the construction of phase diagrams very efficient. See \href{https://github.com/qwcsu/PSCFplus/blob/master/doc/notes/ACAP.pdf}{\texttt{ A\+C\+A\+P.\+pdf}} for more details.  
\item The approach used by P\+S\+CF to solve the S\+CF equations (for an incompressible system) does not allow any athermal species in the system, which has no Flory-\/\+Huggins-\/type interactions with all other species. This problem is solved in P\+S\+C\+F+; see \href{https://github.com/qwcsu/PSCFplus/blob/master/doc/notes/SlvSCF.pdf}{\texttt{ Slv\+S\+C\+F.\+pdf}} for more details.  
\end{DoxyItemize}

P\+S\+C\+F+ is free, open-\/source software. It is distributed under the terms of the G\+NU General Public License (G\+PL) as published by the Free Software Foundation, either version 3 of the License or (at your choice) any later version. P\+S\+C\+F+ is distributed without any warranty, without even the implied warranty of merchantability or fitness for a particular purpose. See the \href{https://www.gnu.org/licenses/gpl-3.0.html\#license-text}{\texttt{ L\+I\+C\+E\+N\+SE file}} or the \href{https://github.com/qwcsu/PSCFplus/blob/master/LICENSE}{\texttt{ gnu web page}} for details.

~\newline
 \mbox{\hyperlink{index}{Main Page}} (Up) ~ ~ ~ ~ \mbox{\hyperlink{install_page}{Installation}} (Next) 